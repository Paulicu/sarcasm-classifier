\section{Related Work}

In this section, I will present notable scientific literature that is relevant to my research questions. 
The existing studies in this area can be broadly categorized into those emphasizing the importance of contextual information, 
those utilizing multi-modal data, and those addressing the subject through various machine learning and deep learning techniques.

The most important piece of literature for our research is the creation of the Self-Annotated Reddit Corpus (SARC) dataset \cite{SARC}, which we will detail further in the following section. 
This work has become very important for sarcasm detection research due to its unprecedented scale and quality. The authors constructed SARC by leveraging Reddit's comment structure and its 
standardized annotation for sarcasm. This approach allowed them to compile a dataset that not only surpasses prior sarcasm datasets in size but also provides rich contextual information for 
each statement. To assess the quality of SARC, they conducted evaluations comparing it to existing datasets, focusing on accuracy and noise levels. They also established performance benchmarks 
for sarcasm detection by testing baseline machine learning models and comparing their results to human performance. The purpose of this research was to create a publicly available dataset that 
could serve as a foundation for future studies, and to provide baseline performance benchmarks that future researchers might build upon and improve.

One notable study \cite{Sandor_2024} that utilized this dataset, implemented several machine learning models, as well as deep learning models based on bidirectional long short-term memory (BiLSTM) 
networks and a model utilizing bidirectional encoder representations from transformers (BERT). The findings indicate that deep learning models, particularly the BERT-based model, outperformed 
traditional machine learning approaches in detecting sarcasm. This suggests that models capable of capturing complex language patterns are more effective at interpreting sarcastic content. 
The authors also discuss potential enhancements for sarcasm detection systems, particularly through various feature extraction techniques for vectorizing the data, which I explore in this paper.

In another approach, \cite{Helal_2024} introduced a contextual-based method for sarcasm detection that emphasizes the importance of surrounding text and the broader context in which comments 
are made. The authors employed pre-trained transformer models, specifically RoBERTa and DistilBERT and fine-tuned them on two datasets, one with textual data and another containing multi-modal 
data (audio, video). By incorporating contextual information, their approach achieved notable F1 scores on both datasets. To enhance efficiency, they experimented with summarizing context into 
concise sentences, which reduced training time considerably. This research proves that significance of context is quite high and yields strong performances on audio or video data. However, 
text-based sarcasm still remains harder to deduce.

Another significant research \cite{Das_2021} explores advanced methodologies for identifying sarcasm in textual data.
Recognizing sarcasm's nuanced nature, which often intertwines humor and mockery, the researchers emphasize the importance of sophisticated models to accurately detect such expressions. 
The authors introduce a novel approach that involves manually extracting sarcastic word distribution features from a benchmark pop culture sarcasm corpus, comprising both dialogues and monologues. 
These features are transformed into weighted input sequences, which are then processed through an ensemble of four parallel deep Long Short-Term Memory (pLSTM) networks, each equipped with 
distinct activation classifiers. For validation, the model was tested on two selected English humor literature pieces from Project Gutenberg. The results obtained surpass previous benchmarks in 
sarcasm detection, setting a new standard for performance in this domain. The successful application of the pLSTM framework highlights its potential for broader use in sentiment analysis and 
natural language processing tasks, especially those involving complex linguistic constructs like sarcasm.
