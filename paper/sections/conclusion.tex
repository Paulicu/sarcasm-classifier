\section{Conclusion}

This study aimed to investigate the effectiveness of various machine learning and deep learning models in detecting sarcasm in textual data. 
Specifically, I explored the performance of Logistic Regression and Ridge Regression using both Term Frequency-Inverse Document Frequency (TF-IDF) and Bag-of-Words (BoW) representations, 
alongside Long Short-Term Memory (LSTM) networks utilizing GloVe and Word2Vec embeddings. My findings indicate that TF-IDF-based models, particularly Logistic Regression with TF-IDF, achieved 
the best balance between accuracy and minimizing misclassifications, while BoW-based models tended to struggle more with detecting sarcasm.

On the deep learning front, the LSTM model using Word2Vec embeddings outperformed the one using GloVe, highlighting the importance of word representation in capturing the nuances of sarcasm. 
Overall, while both traditional and deep learning approaches demonstrated reasonable effectiveness in sarcasm detection, the choice of word embeddings and feature representation has shown 
differences in model performance. Future work may focus on refining these models further by incorporating more advanced techniques and additional contextual information to enhance sarcasm 
detection capabilities.
